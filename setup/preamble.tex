%  A simple AAU report template.
%  2014-09-13 v. 1.1.0
%  Copyright 2010-2014 by Jesper Kjær Nielsen <jkn@es.aau.dk>
%
%  This is free software: you can redistribute it and/or modify
%  it under the terms of the GNU General Public License as published by
%  the Free Software Foundation, either version 3 of the License, or
%  (at your option) any later version.
%
%  This is distributed in the hope that it will be useful,
%  but WITHOUT ANY WARRANTY; without even the implied warranty of
%  MERCHANTABILITY or FITNESS FOR A PARTICULAR PURPOSE.  See the
%  GNU General Public License for more details.
%
%  You can find the GNU General Public License at <http://www.gnu.org/licenses/>.
%
\documentclass[11pt,twoside,a4paper,openright]{report}
%%%%%%%%%%%%%%%%%%%%%%%%%%%%%%%%%%%%%%%%%%%%%%%%
% Language, Encoding and Fonts
% http://en.wikibooks.org/wiki/LaTeX/Internationalization
%%%%%%%%%%%%%%%%%%%%%%%%%%%%%%%%%%%%%%%%%%%%%%%%
% Select encoding of your inputs. Depends on
% your operating system and its default input
% encoding. Typically, you should use
%   Linux  : utf8 (most modern Linux distributions)
%            latin1 
%   Windows: ansinew
%            latin1 (works in most cases)
%   Mac    : applemac
% Notice that you can manually change the input
% encoding of your files by selecting "save as"
% an select the desired input encoding. 
\usepackage[utf8]{inputenc}
% Make latex understand and use the typographic
% rules of the language used in the document.
\usepackage[danish,english]{babel}
% Use the vector font Latin Modern which is going
% to be the default font in latex in the future.
\usepackage{lmodern}
% Choose the font encoding
\usepackage[T1]{fontenc}
% For checkmarks: \cmark and crossmarks: \xmark
\usepackage{pifont}
	\newcommand{\cmark}{\ding{51}}%
	\newcommand{\xmark}{\ding{55}}%
%%%%%%%%%%%%%%%%%%%%%%%%%%%%%%%%%%%%%%%%%%%%%%%%
% Graphics and Tables
% http://en.wikibooks.org/wiki/LaTeX/Importing_Graphics
% http://en.wikibooks.org/wiki/LaTeX/Tables
% http://en.wikibooks.org/wiki/LaTeX/Colors
%%%%%%%%%%%%%%%%%%%%%%%%%%%%%%%%%%%%%%%%%%%%%%%%
% load a colour package
\usepackage[table,dvipsnames]{xcolor}
\definecolor{aaublue}{RGB}{33,26,82}% dark blue
\definecolor{lightGrey}{RGB}{240,240,240}% 
% The standard graphics inclusion package
\usepackage{graphicx}
% Load package to convert eps-files to use as figures
\usepackage{epstopdf}

%\usepackage[dvips,final]{graphicx} 
%\usepackage[dvips]{geometry}
\usepackage{color} %include even if images aren’t in color \usepackage{epsfig}
\usepackage{latexsym}
\usepackage{pstricks}

%\usepackage{epsfig}

% Set up how figure and table captions are displayed
\usepackage{caption}
\captionsetup{%
  font=footnotesize,% set font size to footnotesize
  labelfont=bf % bold label (e.g., Figure 3.2) font
}
% For figures
\usepackage{float}
% For subfigures
\usepackage{subcaption}
% Make the standard latex tables look so much better
\usepackage{array,booktabs}
% Enable the use of frames around, e.g., theorems
% The framed package is used in the example environment
\usepackage{framed}
\usepackage{epstopdf}

% Afstand mellem listepunkter og tilføjelse af resume funktion til lister: \begin{enumerate}[resume]
\usepackage{enumitem}
\setlist{itemsep=-2pt}

% Tilføjer mulighed for at lave enkelte sider i landskab.
\usepackage{lscape}
\usepackage{rotating}

\newcounter{listcounter}
%%%%%%%%%%%%%%%%%%%%%%%%%%%%%%%%%%%%%%%%%%%%%%%%
% Mathematics
% http://en.wikibooks.org/wiki/LaTeX/Mathematics
%%%%%%%%%%%%%%%%%%%%%%%%%%%%%%%%%%%%%%%%%%%%%%%%
% Defines new environments such as equation,
% align and split 
\usepackage{amsmath}
% Adds new math symbols
\usepackage{amssymb}
% Use theorems in your document
% The ntheorem package is also used for the example environment
% When using thmmarks, amsmath must be an option as well. Otherwise \eqref doesn't work anymore.
\usepackage[framed,amsmath,thmmarks]{ntheorem}

% Tilføjer \degree symbol
\usepackage{textcomp}
\usepackage{gensymb}

% Fjerner mellemrum efter komma i formler.
%\usepackage{icomma}

% Packages for SI units
\usepackage[binary-units]{siunitx}
% Format SI units as italic in italic texts
\sisetup{detect-all}
\sisetup{per-mode=symbol}


% Argument til amsmath der gør parenteser uden om parenteser pænere ved brug af \right og \left kommandoerne
\delimitershortfall=-1pt

%%%%%%%%%%%%%%%%%%%%%%%%%%%%%%%%%%%%%%%%%%%%%%%%
% Page Layout
% http://en.wikibooks.org/wiki/LaTeX/Page_Layout
%%%%%%%%%%%%%%%%%%%%%%%%%%%%%%%%%%%%%%%%%%%%%%%%
% Change margins, papersize, etc of the document
\usepackage[
  inner=28mm,% left margin on an odd page
  outer=41mm,% right margin on an odd page
  ]{geometry}
% Modify how \chapter, \section, etc. look
% The titlesec package is very configureable
\usepackage[explicit]{titlesec}
%\titleformat*{\section}{\normalfont\Large\bfseries\color{aaublue}}
%\titleformat*{\subsection}{\normalfont\large\bfseries\color{aaublue}}
%\titleformat*{\subsubsection}{\normalfont\normalsize\bfseries\color{aaublue}}
%\titleformat*{\paragraph}{\normalfont\normalsize\bfseries\color{aaublue}}
%\titleformat*{\subparagraph}{\normalfont\normalsize\bfseries\color{aaublue}}
\usepackage{calc}

% Spacing omkring kapiteloverskrift
\titlespacing*{\chapter}{0pt}{40pt}{50pt}

% Overskrift med stort nummer til venstre og titel til højre
%\newlength\chapnumb
%\setlength{\chapnumb}{1.5cm}
%\titleformat{\chapter}[block]
%{\normalfont\bfseries}{}{0pt}
%{\parbox[b]{\chapnumb}{%
	  %\fontsize{2cm}{0}\selectfont\thechapter}%
  %\parbox[b]{\dimexpr\textwidth-\chapnumb\relax}{%
    %\raggedleft%
    %\hfill{\Huge#1}\\
    %\rule{\dimexpr\textwidth-\chapnumb\relax}{.5pt}}}
%\titleformat{name=\chapter,numberless}[block]
%{\normalfont\bfseries}{}{0pt}
	%{\Huge#1}

% Clear empty pages between chapters
\let\origdoublepage\cleardoublepage
\newcommand{\clearemptydoublepage}{%
  \clearpage
  {\pagestyle{empty}\origdoublepage}%
}
\let\cleardoublepage\clearemptydoublepage

% Change the headers and footers
\usepackage{fancyhdr}
\pagestyle{fancy}
\fancyhf{} %delete everything
\renewcommand{\headrulewidth}{0pt} %remove the horizontal line in the header
\fancyhead[RE]{\color{black}\small\nouppercase\leftmark} %even page - chapter title
\fancyhead[LO]{\color{black}\small\nouppercase\rightmark} %uneven page - section title
\fancyhead[LE,RO]{\thepage} %page number on all pages
% Do not stretch the content of a page. Instead,
% insert white space at the bottom of the page
\raggedbottom
% Enable arithmetics with length. Useful when
% typesetting the layout.

\setlength{\headheight}{14pt}

% Raise penalties for bastards
\widowpenalty=10000
\clubpenalty=10000

%%%%%%%%%%%%%%%%%%%%%%%%%%%%%%%%%%%%%%%%%%%%%%%%
% Table of Contents
% http://en.wikibooks.org/wiki/LaTeX/Bibliography_Management
%%%%%%%%%%%%%%%%%%%%%%%%%%%%%%%%%%%%%%%%%%%%%%%%
% Add additional commands for Table of Contents
\usepackage{bookmark}

{\setcounter{tocdepth}{1}}

% Control of space between items in Table of Contents
\usepackage[titles]{tocloft}
\setlength{\cftbeforepartskip}{10pt}
\setlength{\cftbeforechapskip}{4pt}
\setlength{\cftbeforesecskip}{2pt}
%%%%%%%%%%%%%%%%%%%%%%%%%%%%%%%%%%%%%%%%%%%%%%%%
% Bibliography
% http://en.wikibooks.org/wiki/LaTeX/Bibliography_Management
%%%%%%%%%%%%%%%%%%%%%%%%%%%%%%%%%%%%%%%%%%%%%%%%
% Add the \citep{key} command which display a
% reference as [author, year]
\usepackage[square]{natbib}

%%%%%%%%%%%%%%%%%%%%%%%%%%%%%%%%%%%%%%%%%%%%%%%%
% Misc
%%%%%%%%%%%%%%%%%%%%%%%%%%%%%%%%%%%%%%%%%%%%%%%%
% Add bibliography and index to the table of
% contents
\usepackage[nottoc]{tocbibind}
% Add the command \pageref{LastPage} which refers to the
% page number of the last page
\usepackage{lastpage}
\usepackage[
%  disable, %turn off todonotes
  colorinlistoftodos, %enable a coloured square in the list of todos
  textwidth=\marginparwidth, %set the width of the todonotes
  textsize=scriptsize, %size of the text in the todonotes
  ]{todonotes}

% Add command \includepdf to add a whole pdf page to document
\usepackage{pdfpages}


% Add option to easy format directory tree
\usepackage{dirtree}

% String manipulation
\usepackage{xstring,xifthen}

% Tikz package for drawing nice figures
\usepackage{tikz}

% Package for drawing pretty schematics, without leaving LaTex
\usepackage[american currents, american voltages, european resistors, cute inductors,
american ports]{circuitikz}

% Code syntax highlight
\usepackage{listings}
\lstset{breaklines=true,
		breakatwhitespace=true,
		commentstyle=\color{ForestGreen},
		numbers=left,
		numberstyle=\tiny\color{black},
		keywordstyle=\color{blue},
		basicstyle=\footnotesize\ttfamily,
        showstringspaces=false,
		}
\renewcommand{\lstlistingname}{Code Snippet}

%%%%%%%%%%%%%%%%%%%%%%%%%%%%%%%%%%%%%%%%%%%%%%%%
% Table environments
% http://en.wikibooks.org/wiki/LaTeX/Tables
%%%%%%%%%%%%%%%%%%%%%%%%%%%%%%%%%%%%%%%%%%%%%%%%
% Better table environments for stuff like table width specifier
\usepackage{tabularx}
\usepackage{multirow}
\usepackage{longtable}
%%%%%%%%%%%%%%%%%%%%%%%%%%%%%%%%%%%%%%%%%%%%%%%%
% Project info and abstract
% chapters\abstract.tex, chapters\projectinfo.tex
%%%%%%%%%%%%%%%%%%%%%%%%%%%%%%%%%%%%%%%%%%%%%%%%
% Loads project info and abstract for use in
% hypersetup
\newcommand{\projectFaculty}{%
\iflanguage{english}{%
Electronic Engineering and IT%
}{%
Elektronik og IT%
}}

\newcommand{\projectGroup}{%
\iflanguage{english}{%
Group 18gr872%
}{%
Gruppe %
}}

\newcommand{\projectSemester}{%
P8%
}

\newcommand{\projectType}{%
\iflanguage{english}{%
Project Report%
}{%
Projektrapport%
}}

\newcommand{\projectTitle}{%
\iflanguage{english}{%
Low/Mid Frequency Beamforming%
}{%
Low/Mid Frequency Beamforming%
}}

\newcommand{\projectSubtitle}{%
\iflanguage{english}{%
- Subtitle -%
}{%
- Undertitel -%
}}

\newcommand{\projectTheme}{%
\iflanguage{english}{%
Sound technology for the normal hearing%
}{%
Digitale og analoge systemer i samspil med omverdenen%
}}

\newcommand{\projectPeriod}{%
\iflanguage{english}{%
MSc, 8th Semester 2018%
}{%
Efterårssemester 2016%
}}



\newcommand{\projectParticipants}{%
Jonas Buchholdt\\
Christoph Kirsch
}

\newcommand{\projectSupervisors}{%
Christian Sejer Petersen
}

\newcommand{\projectCopies}{8}

\newcommand{\projectCompletion}{
\iflanguage{english}{%
30th may 2018%
}{
30. may 2018%
}}




\newcommand{\projectAbstract}{
This project deals with low/mid frequency directivity control.
The directional characteristics of a single loudspeaker and position of its acoustic center are characterised. An analytical model of the loudspeaker is established to model a beamforming array. After pointing out some properties of the commonly known first order gradient source, a three speaker array is designed. Three speakers are chosen to overcome the disadvantage of the first order gradient speaker not being able to freely control its main lobe direction. To predict the behaviour of the speaker array in a non-free-field environment, a numerical model is set up.  A genetic algorithm is chosen to optimize gain and phase for the individual loudspeakers. A suitable positioning scheme for the speakers is established. Required signal processing parameters are implemented as filters. The measured directional characteristics of the array are compared with the analytical and numerical models and compared with the directional characteristics of a commercial product.

}

\newcommand{\projectSynopsis}{
Synopsis
}

%%%%%%%%%%%%%%%%%%%%%%%%%%%%%%%%%%%%%%%%%%%%%%%%
% Hyperlinks
% http://en.wikibooks.org/wiki/LaTeX/Hyperlinks
%%%%%%%%%%%%%%%%%%%%%%%%%%%%%%%%%%%%%%%%%%%%%%%%
% Enable hyperlinks and insert info into the pdf
% file. Hypperref should be loaded as one of the 
% last packages
\usepackage{hyperref}
\hypersetup{%
	%pdfpagelabels=true,%
	plainpages=false,%
	pdfauthor={\projectGroup, \projectFaculty, \iflanguage{english}{Aalborg University}{Aalborg Universitet}},%
	pdftitle={\projectTitle},%
	pdfsubject={\projectTheme},%
	bookmarksnumbered=true,%
	colorlinks,%
	citecolor=black,%aaublue,%
	filecolor=black,%aaublue,%
	linkcolor=black,%aaublue,% you should probably change this to black before printing
	urlcolor=black,%aaublue,%
	pdfstartview=FitH,%
	bookmarksdepth=2,%
}

% Defines where URLs should break
\def\UrlBreaks{\do\/\do-\do_}
\urlstyle{same}

% Give the possibility to autoformat reference based on distance to the referenced page. Ex. \vpageref{}
\usepackage{varioref}


% Package to warn about missing references.
%\usepackage{refcheck}

% Package to make a glossary of acronyms.
\usepackage{glossaries}
\glstoctrue
\makenoidxglossaries
% Glossaries package
% http://ctan.cs.uu.nl/macros/latex/contrib/glossaries/glossariesbegin.pdf
%
% % % % % % % %
%	Example of glossary entry:
% \newglossaryentry{cabbage}{name={cabbage},description={vegetable with thick green or purple leaves}}
%
%	Example of acronym entry:
% \newacronym{spi}{SPI}{Serial Peripheral Interface}
%


\newacronym{dgps}{DGPS}{Differential \glsentryshort{gps}}


\newacronym{spl}{SPL}{Sound pressure level}
\newacronym{db}{dB}{Decibel}
\newacronym{hz}{Hz}{Hertz}
\newacronym{ft}{FT}{Fourier Transform}
\newacronym{ift}{IFT}{inverse Fourier Transform}
\newacronym{fft}{FFT}{Fast Fourier Transform}
\newacronym{ifft}{IFFT}{inverse Fast Fourier Transform}
\newacronym{ttl}{TTL}{Transistor–transistor logic}
\newacronym{dll}{DLL}{dynamic link library}
\newacronym{udp}{UDP}{User Datagram Protocol}
\newacronym{ip}{IP}{Internet Protocol}
 \newacronym{dut}{DUT}{Device Under Test}
 \newacronym{usb}{USB}{Universal Serial Bus} 
\newacronym{bandk}{B\&K}{Brüel \& Kjær}
\newacronym{mdf}{MDF}{medium-density fibreboard}
\newacronym{rms}{RMS}{Root Mean Square}
\newacronym{fdtd}{FDTD}{Finite-Difference Time-Domain}
\newacronym{sp}{SP}{Signal Processing}
\newacronym{ga}{GA}{Genetic Algorithm}
\newacronym{dc}{DC}{Direct Current}
\newacronym{fir}{FIR}{Finite Impulse Response}
\newacronym{iir}{IIR}{Infinite Impulse Response}
\newacronym{aau}{AAU}{Aalborg University}
\newacronym{bc}{BC}{Bone Conduction}
\newacronym{ac}{AC}{Air Conduction}
\newacronym{hrtf}{HRTF}{Head Related Transfer Function}
\newacronym{bm}{BM}{Basilar Membrane}
\newacronym{best}{BEST}{Balanced Electromagnetic Separation Transducer}







% Package to make semi-bold font.
\usepackage[outline]{contour}
\contourlength{0.1pt}
\contournumber{50}%
\newcommand{\textsb}[1]{\contour{black}{#1}}

% Package for splitting lists or other things up in columns. Ex: \begin{multicols}{2}
\usepackage{multicol}

% Package for rotating a page to landscape orientation. Ex: \begin{landscape}
\usepackage{pdflscape}

% For adding notes to tables
\usepackage{threeparttable}

% For adding smileys ;) I know....
\usepackage{MnSymbol,wasysym}

% For adding algorithms
\usepackage{algorithm}
\usepackage[noend]{algpseudocode}
\makeatletter
\def\BState{\State\hskip-\ALG@thistlm}
\makeatother

\renewcommand*{\glsgroupskip}{\vspace{2mm}}
% ps to PDF


%Highlight
\usepackage{soul}
\usepackage{amsmath}


\usepackage{pdfpages}

\usepackage{csquotes}

\usepackage{tabularx}
\def\arraystretch{1.5}%

\usepackage{pdfpages}
\usepackage{graphicx}