\section{Introduction}\label{sec:method_intro}
Assessing intelligibility is an essential component of the project at hand. 
In \citep[p. 11]{cote_2011}, the author defines intelligibility as \enquote{the ability to extract the content of the speech signal (\dots) from the recognized phonemes}. Phonemes are commonly defined as the smallest distinctive units of speech.
The approach to acquiring data in order to evaluate intelligibility, may have a substantial influence on the outcome of the assessment.
Recalling from autorefsec:problem_statement, the goal of the investigations is assessing whether there is a difference in intelligibility between \gls{ac} and \gls{bc} speech signals during the presence of ambient noise. 

According to \citep{arl_us_army}, there are three basic priniciples, that are being employed for intelligibility assessment:
\begin{itemize}
\item \textit{Speech intelligibility rating scales}: Listeners grade the perceived quality subjectively using a verbal or numeric scale.
\item \textit{Perceptual intelligibility tests}:  Test signals, which can be comprised of phonemes, syllables, words, sentences or paragraphs, are presented to subjects. The score is evaluated based upon how much information they recognized.
\item \textit{Technical speech intelligibility predictors}: A transmission channel is tested with a specified routine. Parameters like e.g. frequency response or nonlinear behaviour are evaluated in a score, that corresponds to the expected speech intelligibility.
\end{itemize}
Prediction methods are commonly employed for predicting the intelligibility sound through a transmission channel, that can either be a communication system or a \gls{pa} of some kind. \citep{iec_60268} describes the calculation of a widespread measure of transmission quality, that is called the \gls{sti}. 
The prediction approach is unsuitable for the given application, because it relies heavily on assumptions about the human sound perception. Those are established well enough for \gls{ac}, but not for \gls{bc}. As this project is investigating differences in the intelligibility of \gls{ac} and \gls{bc} sound, predicting intelligibility, while relying on the available data for \gls{bc} threshold (\citep{iso_389-3}) and the datasheet of the available bone transducer is unlikely to yield valid results.
Instead, an assessment based on trials with subjects is more promising.
The inherent problem with speech intelligibility rating based scales with subjective parameters is, that they rely on assessment of the judges, intead of objective data. This may be an advantage when it comes to making a transmission pleasant to the recipient, in the context of the current project however, it introduces difficulties.
\citep[Sec. 5.1]{arl_us_army} mentions, that training the listeners has an impact on test results. Also, speech intelligibility ratings can be problematic to compare to speech intelligibility scores from different tests. This complicates comparisons with other studies.
The most suited approach for this project therefore is a perceptual intelligibility test. There are many test methods, some of which will be described in \autoref{sec:methods}. The test can score the intelligibility based on phonemes, syllables, words, sentences or paragraphs. An overview on how to relate scores from tests with phonemes, words and sentences is given in \citep{olsen_1997}

\section{Test Methods}\label{sec:methods}

\subsection{\gls{cat}}\label{ssec:cat}
The \gls{cat} is a speech intelligibility test, that has been developed specifically in context with military applications. The test items are so called callsigns, which are a combination of 18 selected elements from the International Radiotelephony Spelling Alphabet and seven numerical digits, resulting in a total of 126 test items.
Typically, when performing this test, subjects are presented a test item (e.g. Bravo Six) and then type in, what they heard on a computer (e.g. B 6). This procedure is repeated with other test items and the test score is evaluated based on how many items have been identified correctly.
The execution of the test and test results, that have been obtained with different kinds of noise, are described in \citep{rao_2006}.
Contextualising the \gls{cat} with the project at hand, there are benefits and drawbacks. A general advantage of the \gls{cat} is its viability for an automated conducting and scoring due to the type of responses given by the subjects. The nature of the test items makes it possible to get a larger number of responses from a subject in a given time span when compared to sentence based tests.
This is important when designing the test, because of time constraints (see \autoref{sec:test_design}).
The limited set of test items is also paralleled by drawbacks. As the test has been specifically designed for military personnel familiar with the test items, subjects, that do not have this background are likely to perform comparably, unless they are extensively familiarized with the test items, which is time intensive. There is a high likelihood of at least partial repetitions of test items when running the test. If the subjects are not familiarized sufficiently with the test items, a \enquote{learning experience} menaces to pose an uncontrollable influence on the results.

\subsection{\gls{fst}}\label{ssec:fst}
optional maybe
\subsection{\gls{hint}}\label{ssec:hint}
\subsection{\gls{mrt}}\label{ssec:mrt}

