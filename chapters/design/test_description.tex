In order to assess the performance of \gls{bc} and \gls{ac}, an objective test needed to be designed. However, since not all variables could be assessed from an objective perspective, part of this test relies on the subject's perception and the necessary adjustments derived from it.

One key constraint that needed to be taken into account for this test was time, since the procedure itself demands a high level of concentration from the subject and thus couldn't be extended over a long period. Therefore, and after performing some pilot tests, it was decided to limit the total duration to up to \SI{1}{\hour}.
\section{Test Description}
\label{sec:test_description}

As the main focus of this research is the assessment on intelligibility score, the endpoint of it is to compare both types of transducers' performance from the \gls{hint}. However, several previous steps are needed in order to assess the \gls{hint} results correctly. Therefore, the test was divided into three main parts (with an extra step for subject acceptance), and will be referred to as \gls{bier}.

\subsection{Subject acceptance questionnaire}
Previous to accepting possible test subjects, it was necessary to send them a questionnaire according to ISO 389-9:2009REFMISSING?? in order to perform a quick background check and identify possible unfit candidates.
This questionnaire can be found in ANNEXREFMISSING??.

\subsection{Partial audiometry}
The first part of the test consisted of a reduced audiometry in order to ensure normal hearing amongst the participants. As a first approach, a full audiometry was proposed, but was later discarded since it considerably increased the duration of the test and its difficulty, since it demands a lot of focus from the participants. Furthermore, the added data was not of real use for the experiment due to the bandwidth limitation imposed by the bone transducer.
Therefore, a reduced version was selected, containing 500, 750, 1000, 2000, 3000 and \SI{4000}{\hertz}

This part of the test was divided into three subparts, with a little break between them. The duration of the break was subject-dependant and no longer than 30 seconds.
The first subpart consisted on a familiarisation round with two frequencies (\SI{500}{\hertz} and \SI{1}{\kilo\hertz}) through the right ear in order for the participant to fully understand the procedure. Once this part had been completed, the audiometry for both ears was performed, starting with the left one.

\subsection{Level matching}
One of the key factors in the test was to maintain the conditions for both transducers as close as possible so the comparison could be made. However, when designing the test, a challenging point was pinpointed, since it was necessary that the loudness perceived from both transducers matched.

In order to do so, a matching level phase was included in the test. During this phase, the same speech shaped noise was presented to the subject through \gls{ac} and \gls{bc} alternatively every 2 seconds by using the Alternate function within Affinity's software. The level for the \gls{ac} was fixed, as it was measurable, and taken as a reference. The level for the \gls{bc} was started as relatively low in comparison to the \gls{ac} and was modified according to the feedback received from the subject. Every time the noise was transmitted through the \gls{bc}, the subject was instructed to indicate the conductor of the experiment if it needed to be lower, louder or if it was perceived at the same level as the \gls{ac}. Once the subject instructed that the level was perceived the same, to which we will refer as a hit, the conductor would still modify the \gls{bc} until more hits were reached in order to pinpoint the level in a more precise way.
This procedure was repeated in three rounds, first one being familiarisation. Each round would consist on up to 4 hits, depending on the subject's consistency and the conductors assessment. Afterwards, the average between the level of all hits would be taken as the \gls{bc} level to use in the last part of the test.


\subsection{\gls{hint}}
The last part of the \gls{bier} corresponds to the performance of the \gls{hint}. This part is divided into five sub-sections, as it has a familiarisation round and four proper test rounds.

As explained in section \autoref{ssec:hint}, during \gls{hint} the subject is presented 20 sentences contained in a predefined list as well as speech-shaped noise??. For the classic \gls{hint}, both the speech and the noise are presented through the same transducer. However, in order to maintain the same ear canal conditions for the different transducers used in \gls{bier}, the noise was presented through a speaker located inside the room with the subject, and the speech was presented through the headphones or the B81 \gls{bc} transducer. By doing this, the blocked ear canal condition that was present during the headphones use was still maintained while using the \gls{bc}.

During the familiarisation round, half of the sentences were presented through \gls{ac} and the other half through \gls{bc}, while for the test rounds, the whole set of sentences was presented through the same transducer. The noise level was maintained the same for the duration of the whole experiment, both inter-subjects and inter-rounds.

As per danish \gls{hint}'s design, three familiarisation lists are available, as well as ten test lists.