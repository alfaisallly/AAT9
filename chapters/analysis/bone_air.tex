\section{Relation of Airborne and Bone Induced Sound}
For a considerable amount of time there has been an interest in characterizing the way, that \gls{bc}-induced sound is perceived in comparison to the perception of  airborne sound. In order to do so, knowledge about the anatomical and physiological aspects, that contribute to the perception of sound is necessary.
In \citep{bekesy_1932}, the author is successfully cancelling the perception of a bone induced tone on a human subject with airborne sound. In order to do so, phase and amplitude of the airborne sound are manipulated. This led to the assumption, that inducing sound via the bones result in similar patterns on the basilar membrane (see XXXXX) as airborne sound.
In \citep{lowy_1942}, this cancellation of airborne and bone induced sound is investigated in greater detail. The electrical cochlear response is measured on anesthesized cats and guniea pigs, while an airborne tone is cancelled with a bone induced tone in a frequency range from \SIrange{250}{3000}{\hertz}. It is shown, that, when the signal parameters are adjusted for cancellation at one position of the cochlea, the cancellation is also achieved, when measuring on another position of the cochlea. 