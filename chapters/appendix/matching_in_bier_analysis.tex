

\chapter*{The matching analysis in \gls{bier}}
A test was made to evaluated the chosen test signal in \autoref{apend_matching_signals} between \gls{bc} and \gls{ac}. Ti have been chosen that the matching signal for matching \gls{bc} and \gls{ac} is the original noise signal from the \gls{hint} test. It was observed that the average \gls{spl} from the \gls{bc} was \SI{45.3}{\decibel} for matching the \gls{spl} of the \gls{ac}. It was also observed that the standard deviation was \SI{0.58}{\decibel}. The test war based on three subject. To get a more statistical result, the match test is done in a field test with eight subject.

\section*{Materials and setup}
To match the perceived level from the \gls{bc} vs \gls{ac}, the following materials are used:


\begin{table}[H]
\centering
\caption{Equipment list}
\begin{tabular}{l|l|l|l l}
Description         	& Model                                        & Serial-no  						& AAU-no \\ \hline
PC        			 		& MAcbook                                   & W89242W966H  			& -  \\
RME  					& Fireface UFX                             &  23811948 			 	& 108230 \\
ETYM$\bar{O}$TIC RESEARCH     	&   ER4S            & -   									& 02049 \\
Radioear   				&  B81                            & -   									& - \\
Affinity     				& 2.0                            				& -   									& -  \\
Analysis software   & MATLAB \textsuperscript{\textregistered} R2018b & -          & -     
\end{tabular}
\end{table}



\begin{figure}[H]
\centering
\def\svgwidth{\columnwidth}
\input{figures/appendix/match_test.pdf_t}
\caption{Setup for match test of \gls{bc} vs \gls{ac}.}
		\label{fig:appendix:match_meas_system}
\end{figure}

\section*{Test procedure}


\begin{enumerate}
\item The materials are set up as in \autoref{fig:appendix:match_meas_system}.
\item The \gls{bc} is placed on the head according to \autoref{sec:bc_pos} with a rubber hair band.
\item The \gls{ac} is putted intro the ears.
\item The Affinity is set to Match_bone_air mode as according to \autoref{apend:aff_bc_ac_match} 
\item In the match mode the signal is alternating between the \gls{bc} and \gls{ac} automatic with time step of \SI{2}{\second}.
\item The  subject shall reply back if the level shall be lower, higher or if it is matching  the \gls{ac}. It is only the bone that is changed, the air level is fixed.
\item The test will stop after the level convergence to one level. The convergence level is a level that has been matching three times.
\item The test is repeated two or three times.
\end{enumerate}

\section*{Results}

The result for the familiarisation is as following \autoref{apen:match_bier_fam} for all ten subject. 

\begin{table}[H]
\centering
\caption{Familiarisation of match}
\begin{tabular}{l|lll|ll}
Matching in bier \si{\decibel}  &1\textsuperscript{st} trial & 2\textsuperscript{nd} trial & 3\textsuperscript{rd} trial & $\mu$   & $\sigma$ \\ \hline
Subject 1           & 48    & 48    & 51    & 49   & 1.73  \\
Subject 2           & 50    & 51    & 51    & 50.7 & 0.58  \\
Subject 3           & 45    & 48    & 51    & 48   & 3     \\
Subject 4           & 51    & 51    & 50    & 50.7 & 0.58  \\
Subject 5           & 45    & 47    & 45    & 45.7 & 1.15  \\
Subject 6           & 52    & 53    & 53    & 52.7 & 0.58  \\
Subject 7           & 49    & 48    & 49    & 49.7 & 0.58  \\
Subject 8           & 50    & 49    & 48    & 49   & 1     \\
Subject 9           & 45    & 55    & 56    & 52   & 6.08  \\
Subject 10          & 45    & 50    & 47    & 47.3 & 2.52  \\ \hline
Total               &       &       &       & 49.4 & 2.89 
\end{tabular}
\label{apen:match_bier_fam} 
\end{table}

in \autoref{apen:match_bier_fam}  it can be observed that the average and the standard deviation does not differ much from the initial test in \autoref{apend:match_field_init} where the average was \SI{50.2}{\decibel} and the standard deviation was \SI{2.5}{\decibel}. The dynamic ranged chosen in  \autoref{apend:match_field_init} then is proven to hold in this conducted \gls{bier} test.


The result is as following \autoref{apen:match_bier} for ten subject. 
\begin{table}[H]
\centering
\caption{Test result}
\begin{tabular}{lllllllll}
\multicolumn{1}{l|}{Matching in \gls{bier} [\si{\decibel}] }   & 1\textsuperscript{st} trial & 2\textsuperscript{nd} trial & 3\textsuperscript{rd} trial & 4\textsuperscript{th} trial & 5\textsuperscript{th} trial & \multicolumn{1}{l|}{6\textsuperscript{th} trial}                & $\mu$   & $\sigma$ \\ \hline
\multicolumn{1}{l|}{Subject 1}  & 50    & 52    & 50    & 52    & ND    & \multicolumn{1}{l|}{ND} & 51   & 1.15  \\
\multicolumn{1}{l|}{Subject 2}  & 51    & 51    & ND    & ND    & ND    & \multicolumn{1}{l|}{ND} & 51   & 0     \\
\multicolumn{1}{l|}{Subject 3}  & 45    & 50    & 47    & 47    & 49    & \multicolumn{1}{l|}{50} & 48   & 2     \\
\multicolumn{1}{l|}{Subject 4}  & 45    & 50    & 47    & 50    & ND    & \multicolumn{1}{l|}{ND} & 48   & 2.45  \\
\multicolumn{1}{l|}{Subject 5}  & 43    & 42    & 40    & 43    & ND    & \multicolumn{1}{l|}{ND} & 42   & 1.41  \\
\multicolumn{1}{l|}{Subject 6}  & 52    & 53    & 51    & 52    & ND    & \multicolumn{1}{l|}{ND} & 52   & 0.82  \\
\multicolumn{1}{l|}{Subject 7}  & 46    & 48    & 45    & 45    & ND    & \multicolumn{1}{l|}{ND} & 46   & 1.41  \\
\multicolumn{1}{l|}{Subject 8}  & 50    & 50    & 48    & 47    & ND    & \multicolumn{1}{l|}{ND} & 48.8 & 1.5   \\
\multicolumn{1}{l|}{Subject 9}  & 45    & 40    & 47    & 47    & ND    & \multicolumn{1}{l|}{ND} & 44.8 & 3.30  \\
\multicolumn{1}{l|}{Subject 10} & 50    & 47    & 47    & 47    & 51    & \multicolumn{1}{l|}{48} & 48.3 & 1.75  \\ \hline
\multicolumn{1}{l|}{Total}      &       &       &       &       &       & \multicolumn{1}{l|}{}   & 48.1     & 3.02     
\end{tabular}
\label{apen:match_bier} 
\end{table}

In \autoref{apen:match_bier} the subject is asked to do the same as in the familiarisation twice, where the first one half of the result for one test person is done in one trial and the second half of the result for the same test person is done in a second trial, except of one test subject which only have two result. The tendency of the two last matching compare to the familiarisation is that two of the subject with high standard deviation in the familiarisation has lower standard deviation in the real test. The standard deviation is raised a bit in general in the real test compare to the familiarisation, but the standard deviation between subjects seems to be more consistent. 

\section*{Conclusion}



