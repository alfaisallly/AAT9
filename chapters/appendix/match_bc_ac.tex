\chapter*{\gls{bc} vs \gls{ac} matching}
A test was made to design the matching frequency between \gls{bc} and \gls{ac} with only the project writers as test subject. Due to limited time and limited test subject, the matching test is designed to be done with the noise from the \gls{hint} test. While using the noise a problem arise, the transfer function for \gls{bc} and \gls{ac} is not the same. The transfer function differs mostly in the low frequency below \SI{500}{\hertz} and in the high above \SI{2}{\kilo\hertz} and where the \gls{bc} is lower that the \gls{ac}. Therefore the low frequency form the \gls{ac} might work as a masker and the \gls{ac} therefore is higher in level in the frequency of interest. To test this phenomena, the original noise from the \gls{hint} is compared to filtered versions.

\section*{Materials and setup}
To measure the .... on a guitar, the following materials are used:
\begin{itemize}
\item PC
\item \gls{ac}
\item \gls{bc}
\item Afinity
\end{itemize}

%\begin{figure}[htbp!]
%\centering
%\def\svgwidth{\columnwidth}
%\input{figures/appendix/guitar_frequency_test.pdf_tex}
%\caption{Setup for measuring frequency area on a guitar.}
%		\label{fig:appendix:test}
%\end{figure}

\section*{Test procedure}


\begin{enumerate}
\item The materials are set up as in \autoref{fig:appendix:test}.
\item 
\item  
\item  
\item 
\item 
\end{enumerate}

\section*{Results}

%\begin{figure}[htbp!]
%	\centering
%		\includegraphics[width=1\textwidth]{guitar_low_E_neck.pdf}
%		\caption{Measurement of the low E note on the neck pickup.}
%		\label{fig:appendix:low_E_neck}
%\end{figure}

On  \autoref{fig:appendix:low_E_neck} it is seen that the lowest significant frequency is around \SI{80}{\hertz} and the highest significant frequency is around \SI{400}{\hertz}, when playing the low E note on the guitar, using the neck pickup.

