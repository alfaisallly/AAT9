\chapter*{Test a guitars frequency area}
A test was made to get a view of the ....

\section*{Materials and setup}
To measure the .... on a guitar, the following materials are used:
\begin{itemize}
\item Digilent Analog Discovery 2 (Oscilloscope)
\item Fender Squier Classic Vibe Telecaster (Guitar)
\item Digilent Waveforms 2015 (PC - software)
\end{itemize}

\begin{figure}[htbp!]
\centering
\def\svgwidth{\columnwidth}
\input{figures/appendix/guitar_frequency_test.pdf_tex}
\caption{Setup for measuring frequency area on a guitar.}
		\label{fig:appendix:test}
\end{figure}

\section*{Test procedure}


\begin{enumerate}
\item The materials are set up as in \autoref{fig:appendix:test}.
\item 
\item  
\item  
\item 
\item 
\end{enumerate}

\section*{Results}

\begin{figure}[htbp!]
	\centering
		\includegraphics[width=1\textwidth]{guitar_low_E_neck.pdf}
		\caption{Measurement of the low E note on the neck pickup.}
		\label{fig:appendix:low_E_neck}
\end{figure}

On  \autoref{fig:appendix:low_E_neck} it is seen that the lowest significant frequency is around \SI{80}{\hertz} and the highest significant frequency is around \SI{400}{\hertz}, when playing the low E note on the guitar, using the neck pickup.

