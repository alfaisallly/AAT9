\section{Introduction}
Assessing intelligibility is an essential component of the project at hand. 
In \citep[p. 11]{cote_2011}, the author defines intelligibility as \enquote{the ability to extract the content of the speech signal (\dots) from the recognized phonemes}. Phonemes are commonly defined as the smallest distinctive units of speech.
The approach to acquiring data in order to evaluate intelligibility, may have a substantial influence on the outcome of the assessment.
Recalling from autorefsec:problem_statement, the goal of the investigations is assessing whether there is a difference in intelligibility between \gls{ac} and \gls{bc} speech signals during the presence of ambient noise. 

According to \citep{arl_us_army}, there are three basic priniciples, that are being employed for intelligibility assessment:
\begin{itemize}
\item \textit{Speech intelligibility rating scales}: Listeners grade the perceived quality subjectively using a verbal or numeric scale.
\item \textit{Perceptual intelligibility tests}:  Test signals, which can be comprised of phonemes, syllables, words, sentences or paragraphs, are presented to subjects. The score is evaluated based upon how much information they recognized.
\item \textit{Technical speech intelligibility predictors}: A transimission channel is tested with a specified routine. Parameters like e.g. frequency response or nonlinear behaviour are evaluated in a score, that corresponds to the expected speech intelligibility.
\end{itemize}
Prediction methods are commonly employed for predicting the intelligibility sound through a transmission channel, that can either be a communication system or a \gls{pa} of some kind. \citep{iec_60268} describes the calculation of a widespread measure of transmission quality, that is called the \gls{sti}. 
The prediction approach is unsuitable for the given application, because it relies heavily on assumptions about the human sound perception. Those are established well enough for \gls{ac}, but not for \gls{bc}. As this project is investigating differences in the intelligibility of \gls{ac} and \gls{bc} sound, predicting intelligibility, while relying on the available data for \gls{bc} threshold (\citep{iso_389-3}) and the datasheet of the available bone transducer is unlikely to yield valid results.
Instead, an assessment based on trials with subjects is more promising.